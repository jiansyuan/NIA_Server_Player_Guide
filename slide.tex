

\documentclass{beamer}
\usepackage{ctex, hyperref}
\usepackage[T1]{fontenc}

% other packages
\usepackage{latexsym,amsmath,xcolor,multicol,booktabs,calligra}
\usepackage{graphicx,pstricks,listings,stackengine}

\author{\calligra JiansYuan}
\title{我的世界 NIA 服务器}
\subtitle{玩家游戏指南}
\institute{NIA Server Research \& Development Group}
\date{\today}
\usepackage{NIAServer}

% defs
\def\cmd#1{\texttt{\color{red}\footnotesize $\backslash$#1}}
\def\env#1{\texttt{\color{blue}\footnotesize #1}}
\definecolor{deepblue}{rgb}{0,0,0.5}
\definecolor{deepred}{rgb}{0.6,0,0}
\definecolor{deepgreen}{rgb}{0,0.5,0}
\definecolor{halfgray}{gray}{0.55}

\lstset{
    basicstyle=\ttfamily\small,
    keywordstyle=\bfseries\color{deepblue},
    emphstyle=\ttfamily\color{deepred},    % Custom highlighting style
    stringstyle=\color{deepgreen},
    numbers=left,
    numberstyle=\small\color{halfgray},
    rulesepcolor=\color{red!20!green!20!blue!20},
    frame=shadowbox,
}


\begin{document}

\kaishu
\begin{frame} 
    \titlepage
    \begin{figure}[htpb]
        \begin{center}
            \includegraphics[width=0.33\linewidth]{pic/NIA_Server_Logo.pdf}
        \end{center} 
    \end{figure}
\end{frame}

\begin{frame}
    \tableofcontents[sectionstyle=show,subsectionstyle=show/shaded/hide,subsubsectionstyle=show/shaded/hide]
\end{frame}


\section{关于 NIA 服务器} 

\subsection{梦想与现实}
\begin{frame}{什么是 NIA?}
    \begin{itemize}[<+-| alert@+>]
        \item 我们是一个成熟的 MC 服务器创作团队。
        \item 在 2019 年 1 月 28 号那个晚上,梦开始了。
        \item 2020 年中旬,向往开放与梦想的我们迈入国际服领域。
        \item 开源,是为了 MC。\newline 我们公开源代码给更多服主灵感、启发。 \newline \url{https://www.github.com/NIANIANKNIA/NIASERVER-V4/} \\
    \end{itemize}
    %\footnotetext[1]{\cite{NIA_About}}
\end{frame}

\subsection{理想与抱负}
\begin{frame}{公益、原创、非凡}
    \begin{itemize}[<+-| alert@+>]
        \item 自始至终,我们一直以 “公益·原创·非凡” 作为我们服务器的核心发展理念
        \item 公益---我们一直认为,一个好的服务器真正的作用是带给玩家美好记忆,带给玩家快乐的地方,而不是盈利赚钱的地方,所以我们承诺服务器永远不会有任何充值项目!  
        \item 原创---好的服务器并不是一味的将其他人的成果堆叠在一起,我们创作团队深知这一点,所以服务器所使用的地图、大部分插件均由开发团队自制,我们可以保障服务器所有功能体验的一致性。
        \item 非凡---虽然我们不一定做到最好,但我们一定会努力做到最好!
    \end{itemize}
\end{frame}

\section{游玩指南}

\subsection{前置准备}
\begin{frame}{前置准备}
    \begin{itemize}
        \item 请先加服务器主群:724360499
        \item 也可以添加服主QQ:1020317403
        \item 还有一些其他群。
        \begin{itemize}
            \item 服务器审核群:371417932 
            \item 服务器备份群:258515673 
            \item 服务器OP招募群:543363488
        \end{itemize}
    \end{itemize}
\end{frame}

\subsection{安装游戏}
\begin{frame}{安装游戏}
    \begin{itemize}[<+-| alert@+>]
        \item 首先,你应当下载服务器当前对应的基岩版我的世界
        \item 基岩版几乎支持全平台,你可以根据自己的设备前往 Miscrosoft Store(WIN10)、App Store(IOS)、Google Play(Android) 购买相应的我的世界基岩版并下载
        \item 如果你能力有限,没有足够的钱购买我的世界基岩版,我建议你前往 Minebbs 下载相应的破解版(由于某些原因貌似只有安卓可以使用破解版)
        \item 如果没有特殊说明,服务器版本一般为当期最新的正式版
        \item \href{https://mc.minebbs.com/\#/}{\color{purple}{点这里到 Minebbs 下载!}}
    \end{itemize}
\end{frame}

\subsection{加入白名单}
\begin{frame}{加入白名单}
    \begin{itemize}
        \item 加入服务器官方群聊(某些特殊时期服务器可能会开启审核,此时需要您填写一个简单的问卷,提交后才可以获得主群链接),进入服务器群聊之后,请首先发送 \color{purple}{“\#申请白名”} 然后发送您的XboxID,最后发送 \color{purple}{“\#确定”} 即可成功添加白名单!
    \end{itemize}
\end{frame}

\subsection{进入服务器}
\begin{frame}{请开始你的游戏!}
    \begin{itemize}
        \item 根据群公告的ip地址以及端口,点开主页面-游戏-服务器-添加服务器,输入服务器群群公告提供的ip及端口,(名称随便填)点击保存即可,然后直接点击这个服务器就可以加入游玩啦!
        \item 但是,草率的进入服务器并不明智。
        \begin{itemize}
            \item 在游玩之前游览服务器图鉴是一件非常重要的事情!服务器是空岛玩法,了解基本玩法后再前往服务器游玩!
            \item 这里有一切你想要的服务器玩法和资料:\href{https://docs.mcnia.top/zh-CN/Illustrated.html}{\color{purple}{NIA 服务器图鉴}}
        \end{itemize}
    \end{itemize}
\end{frame}

\subsection{遇到问题?}
\begin{frame}{Q? but A!}
    \begin{itemize}[<+-| alert@+>]
        \item 使用移动网络(流量)进入服务器,却显示没有连接互联网?\newline 
        设置中打开“使用移动网络游戏”即可解决
        \item 显示首先应当通过'微软服务器验证'才可以游玩?\newline 
        删掉后台、关闭游戏重新进入,多次失败建议换网
        \item 显示“您未被邀请在此服务器上游玩”?\newline 
        请检查自己白名单添加的XboxID是否完整、正确(区分大小写)
    \end{itemize}
\end{frame}

\section{规章制度}
\begin{frame}{规章制度}
    \begin{itemize}[<+-| alert@+>]
        \item “矩不正,不可以为方;规不正,不可以为圆。”
        \item 制定规矩,旨在维护游戏体验,创造极致游戏氛围。
        \item 见:\href{https://docs.mcnia.top/zh-CN/regulation.html}{\color{purple}{NIA 服务器规章制度v3.1}}
    \end{itemize}
\end{frame}

\section{其他内容}
\begin{frame}
    \begin{itemize}
        \item 还有待补充!
    \end{itemize}
\end{frame}
 
\section{参考文献}
\begin{frame}[allowframebreaks]
    %\bibliographystyle{apalike}
    %\bibliography{ref}
    % \tiny\bibliographystyle{alpha}
    \begin{enumerate}
        \item NIA Group, JiansYuan, "NIA Server Player Guide Beamer(like PPT)", 4/9/2023 [Online Origin] \\
        \url{https://github.com/jiansyuan/NIA_Server_Player_Guide} \\
        PDF: \url{https://github.com/jiansyuan/NIA_Server_Player_Guide/blob/main/slide.pdf}
        \item NIA Group, "About NIA Server", 3/3/2023 [Online] \\
        \url{https://docs.mcnia.com/zh-CN/About.html}
        \item NIA Group, "Server Playing Guide", 4/4/2023 [Online] \\
        \url{https://docs.mcnia.com/zh-CN/guide.html}
        \item NIA Group, "NIA Server Rules \& Regulations Version 3.1", 3/3/2023 [Online] \\
        \url{https://docs.mcnia.com/zh-CN/regulation.html}
        \item NIA Group, laoyali, "NIA Server Illustration ", 2/18/2023 [Online] \\
        \url{https://docs.mcnia.com/zh-CN/Illustrated.html}
        \item NIA Group, NIA Player, "NIA Server Documental Website", \\
        \url{https://docs.mcnia.com/}
    \end{enumerate}
\end{frame}

\begin{frame}
    \begin{center}
        {\Huge\calligra The End}
    \end{center}
\end{frame}

\begin{frame}
    \begin{figure}[htpb] 
        \begin{center}
            \includegraphics[width=0.666\linewidth]{pic/NIA_Server_Logo.pdf}
        \end{center} 
    \end{figure} 
\end{frame}

\end{document}